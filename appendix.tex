\chapter{Fields and curvature}
\label{field_curvature_math}
    
    Gravitation as described by general relativity is complex, so let us start with a simpler theory: electromagnetism. 
The same mathematics that predicts electromagnetic waves (light), which are our means of detecting gravitational waves in LIGO, can then be extended by analogy to gravitational waves.
This derivation is detailed so that the gravitational analogue can be expressed more simply.
Definitions follow Carroll~\cite{Carroll1997}; for a primer in physical mathematics, see Boas~\cite{Boas}.

\section{Mathematical conventions}

    Let us set geometric, natural units where $1/4\pi\epsilon_0 \rightarrow 1$, $c \rightarrow 1$, $G \rightarrow 1$. 
Greek indices indicate four dimensions, Latin indices three, unless specified otherwise.
Assume the Einstein summation convention, e.g., $x_i y^i \equiv \Sigma_{i=0}^3 x_i y^i$
Anti-symmetrization of indices can be indicated by subscripted square brackets (n.b., all differential forms are antisymmetric), symmetrization by parentheses.

Vectors are typical expressed by reference to index, e.g., a vector $v^\mu$, although implicitly a geometrical vector includes its basis vectors, $e^\mu$ such that $\textbf{v} = v^\mu e_\mu$.
Note that when the vector indices are contravariant, the basis vectors are covariant.
Let subscript commas indicate ordinary partial derivatives, i.e., $x_{,i}^j \equiv \partial_i x^j$.
Semicolons indicate covariant derivatives in general relativity, which reduce to commas (partial derivatives) in flat space. 
Suppose we work in a space supplied with a metric tensor, $g_{\mu\nu}$, which acts as a bilinear operator (generalizing the inner product) to produce infinitesimal arclength $ds$ according to $ds^2 = g_{\mu \nu} dx^\mu dx^\nu$ for infinitesimal lengths $dx^\mu$ in the direction of, and dual to, basis vectors $e_\mu$. 
The metric allows index raising and lowering, e.g., $g_{\mu \nu} x^\mu = x_\nu$, $g^{\mu \nu} x_\mu = x^\nu$.

Vectors and matrices cann be indicated by boldface, e.g., $\textbf{v} = v^\mu e_\mu$ is a vector, $\textbf{M} = M^{\mu\nu} e_\mu e_\mu $ is a matrix, whereas differential forms (dual to vectors) can be indicated by sans serif, e.g., $\textsf{v} = v_\mu dx^\mu$ is a 1-form, $\textsf{M} = M_{\mu\nu} dx^\mu \wedge dx^\nu$ is a differential 2-form.
Tensors generalize vectors and differential forms, including both co- and contra-variant components, and can be built from tensor products $\otimes$, such as $T^{\mu\nu}_{\rho\sigma} = (\textbf{M})\otimes (\textsf{N}) = (M^{\mu\nu}) \otimes (N_{\rho\sigma})= (\textbf{u}^\mu \otimes \textbf{v}^\nu) \otimes (\textsf{x}_\rho \otimes \textsf{y}_\sigma)$.

The wedge $\wedge$ is the exterior or Grassman product, the antisymmetric operator on differential forms that generalizes the cross product, e.g., for the wedge of two 1-forms, $\textsf{v} \wedge \textbf{w} = -\textsf{w} \wedge \textsf{v}$.
Generally, for $p$-form $\textsf{A}$, $q$-form $\textsf{B}$, $(A \wedge B)_{\mu_1 \cdots \mu_{p+q}} = (p+q)!/(p!q!) A_{[\mu_1\cdots\mu_p} B_{\mu_{p+1} \cdots \mu_{p+q} ]}$. 
The Levi-Civita symbol is $\epsilon_{\mu_1 \cdots \mu_n}$, in $\mathbb{R}^n$, $+n$ for even permutation, $-1$ for odd, $0$ otherwise.
The Hodge star maps differential $k$-forms in an $n$-dimensional space to $n-k$-forms, e.g., in 3-space, $\star dx = dy \wedge dz$, $\star dy = dz \wedge dx$, $\star dz = dx \wedge dy$, or from a scalar $f$, $\star f = (f) dx \wedge dy \wedge dz$, generally in $n$-space, $(\star \textsf{A})_{\mu_1 \cdots \mu_{n-p}} = 1/(p!) e^{\nu_1 \cdots \nu_p}_{\mu \cdots \mu_{n-p}} \textsf{A}_{\nu_1 \cdots \nu_p}$.

Index raising and lowering creates a homology between vectors and differential one-forms, $e_\mu \leftrightarrow dx^\mu$ (for coordinate-free math, this is implicit).
By itself, $d$ indicates the exterior derivative; $df$ of a scalar (0-form) $f$ is (modulo homology) its gradient $\nabla$, and $d\textsf{a}$ maps a $k$-form to a $(k+1)$-form, but $d^2$ is always 0.
Precisely, $(d\textsf{A})_{\mu_1 \cdots \mu_{p+1}} = (p+1) \partial_{[\mu_1} \textsf{A}_{\mu_2 \cdots \mu_{p+1}]}$. 


The divergence $\nabla \cdot$ of a vector field $\textbf{X}$ is homologously given by $\star d \star \textbf{X}$, the curl $\nabla \times$ by $\star d \textbf{X}$, and the Laplacian $\nabla^2$ (in 3-space) or d'Alembertian $\Box$ (in 4-space) of a scalar function are given by $\star d \star f$.
Covariant exterior derivatives will be noted by $D \textsf{v} = d \textsf{v} + \omega \wedge \textsf{v}$ for a differential form $\textsf{v}$ and a \textit{connection}; this derivative reduces to $d\textsf{v}$ in flat space.

The usual, coordinate basis vectors set $e_\mu = \partial_\mu$.

\section{Electrodynamics}

Modern physical theories are usually described as extremizing an action, $\mathcal{S}$. 
The action is defined as the integral, over space-time, of a Lagrangian density $\mathcal{L}$. 
To integrate, the integral generally requires a metric, $g_{\mu\nu}$.
Writing the determinant of the metric as $|g| = \det(g_{\mu\nu})$, the action to be extremized (setting $\delta \mathcal{S}$ to 0) is

\begin{equation}
\mathcal{S} = \int \mathcal{L} \sqrt{-|g|} d^4 x.
\end{equation} 

The minus sign accounts for the $-1$ in the Minkowski metric, which physically corresponds to the hyperbolic transformation between space and time in special relativity.
Whereas the usual $\mathbb{R}^3$ Euclidean metric below is $g_{ij}$, the Minkowski $\mathbb{R}^{1,3}$ metric is $\eta_{\mu\nu}$:

\begin{eqnarray}
g_{ij} = \mathbb{I}^3 = 
\left[
\begin{array}{ccc}
1 & 0 & 0 \\
0 & 1 & 0 \\
0 & 0 & 1
\end{array} \right],
\textup{ }
\eta_{\mu\nu} =
\left[
\begin{array}{cccc}
-1 & 0 & 0 & 0\\
0 & 1 & 0 & 0 \\
0 & 0 & 1 & 0\\
0 & 0 & 0 & 1
\end{array} \right].
\end{eqnarray}

Given a Minkowski metric $\eta_{\mu' \nu'}$, a curved 4-space $\mathbb{R}^{1,3}$ metric can be written in \textit{vielbein} terms: $e^{\mu'}_{\mu}$, $e^{\nu'}_{\nu}$ such that $g_{\mu\nu} = e^{\mu'}_{\mu} e^{\nu'}_{\nu} \eta_{\mu' \nu'}$.


The question then becomes how we can define the Lagrangian density for different physical theories. 
In both electromagnetism and gravitation, the quantity is a kind of curvature, contracted from respectively the Maxwell and Ricci tensors. 
This curvature, analogous to an energy term, is then extremized subject to a conserved source term: 4-current in electromagnetism, stress-energy and the cosmological constant in general relativity.

As educated coordinate transformations make a great number of problems simpler -- a theme of the both the feedforward regression and the data analysis presented in later chapters of this thesis -- let us proceed with the derivation in a way that is manifestly covariant under transformations. If desired, one can then extend this treatment to electromagnetism in curved spacetime, such as under the influence of a gravitational wave. 

    Electromagnetism is theoretically defined by a 1-form $\textsf{A} = A_\mu d x^\mu$ where $A^\mu$ is a vector potential~\cite{MisnerThorneWheeler} given by $A_\mu = (\phi, -A_x, -A_y, -A_z)$~\cite{GriffithsE}. 
The scalar electric potential is $\phi$ and the magnetic vector potential is $A_i$.
In a given gauge $\phi$, the theory is \textit{gauge symmetric} ($\phi$ invariant) such that the addition of the differential $d \phi$ to $\textsf{A}$ does not change the theory. 
This changeless symmetry, in the group $U(1)$, arises because the physics of the theory can be described through the derived Maxwell field tensor $F_{\mu \nu}$, or curvature form $\textsf{F}$:

\begin{eqnarray}
\textsf{F} = d \textsf{A} = (\partial_\mu A_\nu) dx^\mu \wedge dx^\nu = \frac{1}{2} F_{\mu\nu} dx^\mu \wedge dx^\nu, \\
F_{\mu \nu} = A_{\nu,\mu} - A_{\mu,\nu},
\end{eqnarray}

By the usual flat-space definitions of potential for an electric field $E$ and magnetic field $B$, $E_i = -\phi_{,i} - A_{i,0}$, $B_i = \epsilon_{ijk} \partial^j A^k = \epsilon_{ijk} A^{k, j}$. 
Electrodynamics in curved space are well-studied; many equations can be translated by replacing partial with covariant derivatives.
Therefore, the Maxwell tensor in explicit coordinate form is

\begin{equation}
F_{\mu\nu} =
\left[
\begin{array}{cccc}
0 & E_x & E_y & E_z\\
-E_x & 0 & -B_z & B_y \\
-E_y & B_z & 0 & -B_x\\
-E_z & -B_y & B_x & 0
\end{array} \right].
\end{equation}

\noindent Maxwell's theory then corresponds to the minimization of the action, finding $\delta S = 0$,


\noindent for a Lagrangian $\mathcal{L}$ with a stress-energy term for the field itself constrained by an interaction with the current 1-form $\textsf{J} = J_\mu d x^\mu$:

\begin{eqnarray}
%\mathcal{L} = - \frac{1}{4} F^{\mu \nu} F_{\mu \nu} + J_\mu A^\mu,\\ 
\mathcal{S} = \int \left( -\frac{1}{4} F^{\mu \nu} F_{\mu \nu} + J_\mu A^\mu \right) \sqrt{-|g|}d^4 x.
\end{eqnarray}

A Hodge dual $\star$ then defines the dual to the Maxwell form, $\textsf{G} \equiv \star \textsf{F}$. 
Maxwellian electromagnetism then is concisely described by extremizing the action to derive the following equations:

\begin{eqnarray}
d \textsf{F} = 0,\\
d \star \textsf{F} = 4 \pi \star \textsf{J}.
\end{eqnarray}  

Applied in flat space, where $g_{\mu \nu} = \eta_{\mu \nu}$, and substituting in the explicit coordinate form of the Maxwell tensor, the theory is easily expressed as four familiar first-order differential equations of the $\textbf{E}$ and $\textbf{B}$ fields in terms of electrical charge density $\rho$ and 3-current $J^i$ (cf. the textbook by Jackson~\cite{JacksonEM}), in (1 time)+(3 space)-dimensions:
 
%(CHECK SIGNS AND UNITS AND FONTS)

\begin{eqnarray}
\star d \star \textbf{E} = E^i_{,i} = \rho,\\
\star d \textbf{E} + \partial_t \textbf{B} = \epsilon^{ijk} E^{k,j} + B^i_{,0} = 0,\\
\star d \star \textbf{B} = B^i_{,i} = 0,\\
\star d \textbf{B} - \partial_t \textbf{E} = \epsilon^{ijk} B^{k,j} - E^i_{,0} = 4 \pi J^i.
\end{eqnarray}

Returning to four full dimensions and specifying the \textit{Lorenz gauge}, where we choose any vector potential for which $\star d \star \textsf{A} = A^\nu_{,\nu} \equiv 0$, 

\begin{eqnarray}
\star d \star d \textsf{A} = -4\pi \textsf{J},
%g^{\mu\nu}A^\sigma_{,\mu\nu} = -4\pi J^\sigma,
\end{eqnarray}
        
\noindent with $\Box \equiv \partial_\mu \partial^\mu$, this is often written as $\Box A^\sigma = -4 \pi J^\sigma$ -- a wave equation. 
This derivation has been laborious in order to establish how differential forms allow a cleaner expression of electromagnetism; the differential form equation also is covariant, accurate in curved spacetime.
Our task now is to compare with general relativity to see whether a similar wave equation will emerge.
