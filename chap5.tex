


        

        \section{Directed TwoSpect}
        \label{directed}

            TwoSpect improvements myself (to do).

            \subsection{Target, directed and all-sky search sensitivity}
            \label{tradeoffs}

                Targeted (known object) vs directed (region)vs all-sky (everything).

            \subsection{Enhancements enabled by directed searching}
            \label{directed_enhancements}

                Modifications for directed search.

        \section{Quantifying Directedness}
        \label{quant_directed}



        
            Quantify how good the improvements are in directed TwoSpect. 

            Cite Feldman-Cousins confidence intervals paper~\cite{FeldmanCousins1998}.

            Cite Ethan's thesis~\cite{RomeroThesis} and our paper, maybe by using generation functions to make better TwoSpect statistics.

        \section{Scorpius X-1 and XTE J1751-305: Directed TwoSpect results}
        \label{directed_results}
 
            Preliminary results of a directed search (possibly simulation-only).

            We have a great deal of material here already, just need to pull from figures and commentary from the Sco X-1 wiki. Keith recommends paralleling the Sco X-1 paper.




        %---------------------------------

	%The following is an example of using the commands \textit{ref}
	%and \textit{label}. With these commands theorems, chapters,
	%sections and figurres can be labeld with names in the tex file
	%and then refered to by these names in later tex files. In
	%chapter~\ref{intro} we saw section~\ref{sample_section} or
	%theorem~\ref{sample_theorem}.

	%Lastly, here is how to include a figure. First generate an
	%encapsulated postscript file in xfig, adobe illustrator or
	%some other program. The specific commands are found in
	%\textit{chap2.tex}.

        %\begin{figure}[htb]
        %\centerline{ \epsfig{figure=sample.eps, 
        %height =  1.5 in}}
        %\caption{Sample Figure}
        %\label{sample_figure}
        %\end{figure}

