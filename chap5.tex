


        

        \section{Directed TwoSpect}
        \label{directed}

            TwoSpect improvements myself (to do).

            \subsection{Target, directed and all-sky search sensitivity}
            \label{tradeoffs}

                Targeted (known object) vs directed (region)vs all-sky (everything).

            \subsection{Enhancements enabled by directed searching}
            \label{directed_enhancements}

                Modifications for directed search.

        \section{Quantifying Directedness}
        \label{quant_directed}

            How big are the $X$ patterns in the $R$ surface on the modulation depth versus frequency plane?
These $X$-patterns follow the lines $df = f_{\textup{signal}} \pm (f - f_\textup{signal})$, intersecting at $f_\textup{signal}$.

We expect the $X$ pattern to be where the template touches a peak or trough in the time-frequency plane, with an $R$ proportional to the peak in a way that is as the relative power of that bin compared the power in the whole sinusoid. 
Since the template touches only once per period, the ratio would be $T_{\textup{coh}} / P_{\textup{signal}}$. For Scorpius X-1 ($P = 68023.8259$ s), that would lead to ratios of roughly $0.026 \approx 1/38$ for 1800 s SFTs, $0.01 \approx 1/81$ for 840 s SFTs, and $0.0053 \approx 1/189$ for 360 s SFTs.

Yet in practice, as exemplified by pulsar 40 in the Mock Data Challenge, which used 360 s SFTs, the difference between the peak (correct template) $R$-value and the $X$-pattern arms was only a factor of 1/2 to 1/3.
This suggests that the `touch-at-one-point' model is oversimplified, and there is significant overlap still.
There would be a shearing process as $df$ deviates further from the true modulation depth, governed by the frequency bin resolution, $1/T_\textup{coh}$. 
For pulsar 40, $df = 0.046$ Hz, the amplitude of this modulation depth was only about $B = \left(2 \pi \textup{asini} f / P \right) \times \textup{coh} \approx $ 16 bins, with a period of 189 bins.
We could estimate the fractional power in the arms to then be $2 \pi / \cos^{-1} ((B-1) / B)$, or roughly $1/18$ for pulsar 40.
Taking the arccosine of $(B-2)/B$ instead yields roughly $1/12$.
This is closer to the 1/2 to 1/3 ratios observed in the analyses that than 1/189 predicted naively, although a proper study would require a closer look at the generation of TwoSpect \textit{exact templates}.

The deeper problem is that the wrong putative template and the true signal are both stationary are the same point in the time-frequency plane, stationary in the sense that their derivatives match:

\begin{equation}
\frac{df_\textup{true} (t)}{dt} = \frac{df_\textup{template}(t)}{dt} = 0.
\end{equation}

Understanding the $X$ structure more fully could inform future methods for maximum likelihood estimation on the $R$ statistic plane or, more likely, on the pushed-forward $p-value$ plane.
Theoretically, combining information from across the parameter space plane should create a strong signal.
This approach could parallel the deconvolution of point spread functions in optical imaging.
Although the deconvolution or Green's function would be different, this might also enhance the angular resolution when applied to spread on the right ascension, declination plane.
Constructing the deconvolved plane should in principle yield the same information as constructing a likelihood surface.
However, this approach needs to be tested.

        
            Quantify how good the improvements are in directed TwoSpect. 

            Cite Feldman-Cousins confidence intervals paper~\cite{FeldmanCousins1998}.

            Cite Ethan's thesis~\cite{RomeroThesis} and our paper, maybe by using generation functions to make better TwoSpect statistics.

        \section{Scorpius X-1 and XTE J1751-305: Directed TwoSpect results}
        \label{directed_results}
 
            Preliminary results of a directed search (possibly simulation-only).

            We have a great deal of material here already, just need to pull from figures and commentary from the Sco X-1 wiki. Keith recommends paralleling the Sco X-1 paper.




        %---------------------------------

	%The following is an example of using the commands \textit{ref}
	%and \textit{label}. With these commands theorems, chapters,
	%sections and figurres can be labeld with names in the tex file
	%and then refered to by these names in later tex files. In
	%chapter~\ref{intro} we saw section~\ref{sample_section} or
	%theorem~\ref{sample_theorem}.

	%Lastly, here is how to include a figure. First generate an
	%encapsulated postscript file in xfig, adobe illustrator or
	%some other program. The specific commands are found in
	%\textit{chap2.tex}.

        %\begin{figure}[htb]
        %\centerline{ \epsfig{figure=sample.eps, 
        %height =  1.5 in}}
        %\caption{Sample Figure}
        %\label{sample_figure}
        %\end{figure}

