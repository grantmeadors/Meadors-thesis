
    \section{Prototypes: travelling kiosks and the Ann Arbor Hands-On Museum} 
    \label{prototypes}

        Prototypes: Ann Arbor Hands-On Museum and travelling kiosk.

    \section{World Science Festival interferometer manufacture}
    \label{manufacture}

        World Science Festival interferometer in isolation.

        \subsection{Laser, optics and display}
        \label{laser_display}

            Laser and optics (and display).

        \subsection{Aluminum baseboard}
        \label{baseplate}

            Aluminum base plate.

        \subsection{Plexiglass enclosure}
        \label{enclosure}

            Plexiglass, many lessons learned.

    \section{Exhibitions: New York City, Portsmouth, Fort Wayne}
    \label{exhibitions}

        World Science Festival interferometer installed.

        \subsection{Exhibit overview}
        \label{exhibit_overview}

            NYC exhibit overview: design, walls, kiosks, displays, interactivities.

        \subsection{World Science Festival 2010}
        \label{WSF2010}

            Success in WSF 2010.

        \subsection{Portsmouth and Fort Wayne}
        \label{secondary_installations}

            Secondary installations and future outreach potential.

    \section{Future LIGO outreach}
    \label{future_outreach}

        Future LIGO outreach? How to explain a new astronomy.


        --------------------------------------

	%The following is an example of using the commands \textit{ref}
	%and \textit{label}. With these commands theorems, chapters,
	%sections and figurres can be labeld with names in the tex file
	%and then refered to by these names in later tex files. In
	%chapter~\ref{intro} we saw section~\ref{sample_section} or
	%theorem~\ref{sample_theorem}.

	%Lastly, here is how to include a figure. First generate an
	%encapsulated postscript file in xfig, adobe illustrator or
	%some other program. The specific commands are found in
	%\textit{chap2.tex}.

        %\begin{figure}[htb]
        %\centerline{ \epsfig{figure=sample.eps, 
        %height =  1.5 in}}
        %\caption{Sample Figure}
        %\label{sample_figure}
        %\end{figure}

