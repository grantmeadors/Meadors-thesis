


 
    \section{Cycles of science}
    \label{cycles}

        How it all fits together.

        \subsection{Improvements to observatories}
        \label{observatories_better}

            Enhancements like enhanced/advanced LIGO and squeezing.

        \subsection{Understanding instruments}
        \label{instrumental_understanding}

            ....necessitate detector characterization, like scans and filters

        \subsection{Refining data}
        \label{data_refinements}

            ....automated feedforward filters yield own enhancements

        \subsection{Searching deep-space}
        \label{searching_space}

            ...TwoSpect and other searches benefit

        \subsection{Reaching out, looking up}
        \label{reaching_out}

            ...Outreach makes research accessible to public.

    \section{Scientific merit: filtering and analysis}
    \label{merit}

        Core projects.

        \subsection{Feedforward improvement to LIGO data}
        \label{feedforward_end}

            Evaluate success of feedforward.

        \subsection{TwoSpect directed search for neutron stars in binary systems}
        \label{TwoSpect_end}

            ...and TwoSpect-directed.

            One thought that might develop into something more fruitful is as follows.

            Someday deconvolve, maybe Bayesian, the skymaps and parameter space spread of TwoSpect with simulation to understand what we are really seeing. Cannot do all templates in paramter space, but only need a few to compare -- in a way, it already does.

    \section{Entering the advanced detector era}
    \label{advanced_detector_era}

        Advanced LIGO: how much better can we do?

    \section{Vision of a dark sky}
    \label{dark_sky}

        Why GW astronomy at all? What could be out there?

%        -------------------------------------
%
%	The following is an example of using the commands \textit{ref}
%	and \textit{label}. With these commands theorems, chapters,
%	sections and figurres can be labeld with names in the tex file
%	and then refered to by these names in later tex files. In
%	chapter~\ref{intro} we saw section~\ref{sample_section} or
%	theorem~\ref{sample_theorem}.
%
%	Lastly, here is how to include a figure. First generate an
%	encapsulated postscript file in xfig, adobe illustrator or
%	some other program. The specific commands are found in
%	\textit{chap2.tex}.
%
%        \begin{figure}[htb]
%        \centerline{ \epsfig{figure=sample.eps, 
%        height =  1.5 in}}
%        \caption{Sample Figure}
%        \label{sample_figure}
%        \end{figure}

