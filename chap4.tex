


        
        \section{Neutron stars in binary systems}
        \label{binary_NS}

            Astrophysical prospects for binary pulsar detection. 
            Binary pulsars are perhaps our best hope for detecting continuous gravitational waves.

            \subsection{Binary spin-up and detectable lifetime}
            \label{spin-up}
         
                GW pulsar lifetime alone vs companion.

            \subsection{Detection rate projections}
            \label{rate_projections}

                aLIGO rate projections.

        \section{TwoSpect all-sky searches}
        \label{all-sky}

            TwoSpect methods as-is. These are described in detail in Evan Goetz's thesis~\cite{GoetzThesis}. Note that the code is located on the web freely accessible in the LALApps repository~\cite{LALAPPSrepo}.

            \subsection{Two spectra: a double Fourier transform}
            \label{two_spectra}

                'Two spectra' -- FFT of periodograms reveals modulation of sine waves.

            \subsection{Infering neutron stars with companions}
            \label{inference}
 
                Infer whether modulation is due to a companion star.

        \section{Scorpius X-1 and results from Directed TwoSpect}
        \label{directed_results}
 
            Preliminary results of a directed search (possibly simulation-only).

            We have a great deal of material here already, just need to pull from figures and commentary from the Sco X-1 wiki. Keith recommends paralleling the Sco X-1 paper.

        %---------------------------------

	%The following is an example of using the commands \textit{ref}
	%and \textit{label}. With these commands theorems, chapters,
	%sections and figurres can be labeld with names in the tex file
	%and then refered to by these names in later tex files. In
	%chapter~\ref{intro} we saw section~\ref{sample_section} or
	%theorem~\ref{sample_theorem}.

	%Lastly, here is how to include a figure. First generate an
	%encapsulated postscript file in xfig, adobe illustrator or
	%some other program. The specific commands are found in
	%\textit{chap2.tex}.

        %\begin{figure}[htb]
        %\centerline{ \epsfig{figure=sample.eps, 
        %height =  1.5 in}}
        %\caption{Sample Figure}
        %\label{sample_figure}
        %\end{figure}

